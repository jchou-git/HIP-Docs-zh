\chapter{概覽}

過去 40 年來,我們看到處理能力有了驚人的進步。微處理器設計師透過增加更多的 transistors 和提高 processor clock 的速度,利用 silicon 技術中的 Moore's Law 和 Dennard scaling,不斷推出性能更強的晶片。然而,在 21 世紀初,正如 IBM 的 Robert Dennard 預測的那樣,晶片的 clock frequency 達到了一個極限。這讓我們無法繼續提高 silicon 的功率密度,因為累積的能量會導致散熱變得非常困難。

為了解決這個問題,晶片製造商開始尋求在單一晶片上加入多個 cores 的 parallel processing 技術。雖然這讓晶片性能有了大幅提升,但多數現有軟體仍是基於 sequential processing 模型設計的,這給程式設計師帶來了新的挑戰,他們需要尋找更創新的方法來在應用中實現 parallelism。

最近,我們看到單一 microprocessor 的 cores 數量從幾個增加到十幾甚至幾十個。例如,AMD 的第三代 Ryzen Threadripper CPU 有多達 64 cores,而下一代預計將達到 128 cores。應用程式設計師開始利用 manycore CPUs 的優勢,因為它們在執行多個同時運行的 sequential threads 上非常強大。

另一個值得注意的趨勢是 heterogeneous computing,這是一種使用專為特定任務設計的硬體平台。例如,顯示卡供應商如 ATI 和 NVIDIA 率先推出了專為資料並行和圖形密集型工作負載設計的 GPUs。這些 GPUs 的特殊設計讓它們性能強大,但也要求開發者使用專有的 graphics languages 撰寫程式,這對於它們作為加速器的廣泛應用造成了一定阻礙。

\section{平行程式設計}
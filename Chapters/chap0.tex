\chapter{ch0}
just a sample chapter

\section{sample}
	\begin{verbatim}
		\begin{The}		\lipsum[1]		\end{The}
	\end{verbatim}
\begin{The}		\lipsum[1]		\end{The}

	\begin{verbatim}
		\begin{Proof}		\lipsum[1]		\end{Proof}
	\end{verbatim}
\begin{Proof}	\lipsum[2]		\end{Proof}

\begin{verbatim}
	\begin{Pro}		\lipsum[1]		\end{Pro}
\end{verbatim}
\begin{Pro}		\lipsum[3] 		\end{Pro}

\begin{verbatim}
	\begin{Le}		\lipsum[1]		\end{Le}
\end{verbatim}
\begin{Le}			\lipsum[5]		\end{Le}

\begin{verbatim}
	\begin{Co}		\lipsum[1]		\end{Co}
\end{verbatim}
\begin{Co}		\lipsum[6]		\end{Co}
%%%%%%%%%%%%%%%%%%%%%%%%%%%%%%%%%%%
\section{Definizioni, esempi, osservazioni}
\begin{verbatim}
	\begin{De}		\lipsum[1]		\end{De}
\end{verbatim}
\begin{De}		\lipsum[7]		\end{De}

\begin{verbatim}
	\begin{Exa}		\lipsum[1]		\end{Exa}
\end{verbatim}\begin{Exa}\lipsum[8] \end{Exa}

\begin{verbatim}
	\begin{Oss}		\lipsum[1]		\end{Oss}
\end{verbatim}
\begin{Oss}\lipsum \end{Oss}
%%%%%%%%%%%%%%%%%%%%%%%%%%%%%%%%%%%
\section{Description, itemize, enumerate}
\paragraph*{} \underline{Description}
\begin{verbatim}
	\begin{description}		\item[Oggetto 1] Prova \item[Oggetto 2] Prova		\end{description}
\end{verbatim}
\begin{description}
	\item[Oggetto 1] Prova 
	\item[Oggetto 2] Prova 
\end{description}

\begin{verbatim}
	\begin{itemize}		\item Prova \item Prova \item Prova		\end{itemize}
\end{verbatim}
\paragraph*{} \underline{Itemize}
\begin{itemize}
	\item Prova 
	\item Prova
	\item \lipsum[9]
\end{itemize}

\begin{verbatim}
	\begin{enumerate}		\item Prova \item Prova		\end{enumerate}
\end{verbatim}
\paragraph*{} \underline{Enumerate}
\begin{enumerate}
	\item Prova
	\item Prova
\end{enumerate}
%%%%%%%%%%%%%%%%%%%%%%%%%%%%%%%%%%%
\section{Note a margine}
	\lipsum[1]
	\note{Nota a margine: può essere utile come segnaposto per definizioni, teoremi... Si usa col comando \text{$\setminus$note\{\#1\}}}
	\lipsum[1]
	\lipsum[1]
		\expl{Ho impostato una versione con un colore diverso per spiegare passaggi nelle dimostrazioni. Si usa col comando \text{$\setminus$expl\{\#1\}}}
	\lipsum[1]
%%%%%%%%%%%%%%%%%%%%%%%%%%%%%%%%%%%
\section{Collegamenti ipertestuali, citazioni, indici, url}
	\paragraph*{} Un collegamento ipertestuale si realizza con
\begin{verbatim}\label{Label}\end{verbatim}		
e si richiama con
 \begin{verbatim}\ref{LabelTesto}\end{verbatim}
 Risultato: \label{LabelTesto}\ref{LabelTesto}

\paragraph*{} Una citazione a un libro in bibliografia si realizza con
\begin{verbatim}\cite{LabelLibro}\end{verbatim}
	Risultato: \cite{Label}

\paragraph*{} Si inserisce un nuovo elemento nell'indice analitico con i comandi
\begin{verbatim}\index{Index}\end{verbatim}
\begin{verbatim}\index{Index!SecondaVoce}
\end{verbatim}
Per il risultato di questi comandi vedere l'indice analitico a fine documento. \index{Index}\index{Index!SecondaVoce}\index{Index!SecondaVoce!TerzaVoce}
\paragraph*{} Per mettere un link si può usare
\begin{verbatim}\url{https://poisson.phc.dm.unipi.it/~puddu}\end{verbatim}
Risultato: \url{https://poisson.phc.dm.unipi.it/~puddu}
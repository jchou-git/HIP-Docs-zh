\documentclass[10pt, twoside]{book}
    \usepackage[italian]{babel}
    \usepackage[utf8]{inputenc}
    \usepackage[T1]{fontenc}
%%%%% Pacchetti
\usepackage{FileAusiliari/Layout}			% Contiene i pacchetti e le impostazioni per il layout
\usepackage{FileAusiliari/Pacchetti}		% Pacchetti aggiuntivi di vario tipo (senza tikz)
\usepackage{FileAusiliari/TikZ}				% Ambiente tikzpicture
\usepackage{FileAusiliari/Definizioni}		% Definizioni di colori, variabili globali ecc.
\usepackage{FileAusiliari/Environments}		% Impostazioni TOC, bibliografia e indice analitico + environments vari per il contenuto del documento
\usepackage{FileAusiliari/Custom}			% Tutto ciò che è personalizzabile normalmente dall'utente (tranne i colori per collegamenti ipertestuali, citazioni, link, che sono da modificare in Referencing)
%%%%%%%%%% Impostazioni indice analitico
%%%%%%%%%%%%%%%%%%%%%%%%%%%%%%%%%%%%
\usepackage{imakeidx}
\indexsetup{othercode=\small}
\makeindex[columns=2, intoc=false, columnseprule, title={}, 
           options= -s FileAusiliari/Stile.ist]
%%%%%%%%%%									Collegamenti ipertestuali
%%%%%%%%%%
\RequirePackage[breaklinks, colorlinks=true, hypertexnames=true, linktoc=all]{hyperref}

	%%%%%%%%%% Colori dei link ipertestuali
	\hypersetup{colorlinks, %I colori sono per il pdf, per la stampa impostare 'black' per tutti i colori
			urlcolor={Url},  % default='Url'
			citecolor={Cite}, % default='Cite'
			linkcolor={Link}} % default='Link'

%%% Miglior referencing
\pagenumbering{none}		% Collegamenti ipertestuali e indice analitico
\usepackage{FileAusiliari/Comandi}			% Comandi vari
\usepackage{fancyhdr}
\usepackage{titlesec}
\usepackage{listings}
\usepackage{graphicx}
\usepackage{caption}
\usepackage{hyperref}
\usepackage{makeidx}
\usepackage{xcolor}


\usepackage{fontspec} % 字體
\usepackage{xeCJK}    % 中文支持
\setmainfont{Times New Roman} % 主字體
\setCJKmainfont{Noto Sans CJK TC} % 中文黑體 思源黑體
\linespread{1.5}

\newcommand{\CJKsection}[1]{{\CJKfamily{titlefont}#1}} % 中文支持 Section Title
\titleformat{\section}{\Large\bfseries\CJKsection}{\thesection}{1em}{}

% 自定義命令
\newcommand{\code}[1]{\texttt{#1}}
\newcommand{\term}[1]{\textit{#1}}
\renewcommand{\bold}[1]{\textbf{#1}}
\newcommand{\figref}[1]{\textbf{Figure~\ref{#1}}}
\newcommand{\tabref}[1]{\textbf{Table~\ref{#1}}}
\newcommand{\lstref}[1]{\textbf{Listing~\ref{#1}}}
\newcommand{\chapref}[1]{\textbf{Chapter~\ref{#1}}}
\newcommand{\secref}[1]{\textbf{Section~\ref{#1}}}



%%%%%%%%%%%%%%%%%%%%%%%%%%%%
%%%%%%%%%%%%%%%%%%%%%%%%%%%%
\begin{document}
%%%%%%%%%%%%%%%%%%%%%%%%%%%%									 TITOLO
%%%%%%%%%%%%%%%%%%%%%%%%%%%%
% \input{FileAusiliari/Titolo}
\begin{titlepage}

% % LOGO
% \begin{figure}[ht]\centering
%     \includegraphics[scale=.6]{FileAusiliari/Logo/MarchioBlack.pdf}
% \end{figure}

% TITLE AND SUBTITLE
\vspace{2.5cm}
\parbox[l]{.9\textwidth}{\centering
    {\HUGE \bfseries 使用 HIP 的加速運算}\\[2\baselineskip]
    {\Large \textit{加速運算技術與應用研究}}\\[.5\baselineskip]}

\vspace*{\fill}

% AUTHOR
\parbox[b]{.5\textwidth}{
    \rule{1pt}{.125\textheight}
    \hspace{0.025\textwidth}
    \parbox[b]{.8\textwidth}{
        {\Large \bfseries 作者}\\[1\baselineskip]
        {\Large \textsc{Yifan Sun, Sabila Al Jannat, \\ Trinayan Baruah, and David Kaeli}}\\[2\baselineskip]
        {\Large 2024 年 10 月}
    }
}

\end{titlepage}

%%%%%%%%%%%%%%%%%%%%%%%%%%%%									FRONTMATTER
%%%%%%%%%%%%%%%%%%%%%%%%%%%%
\frontmatter
\pagestyle{fancyfront}
%%%%%								 INDICE
\begingroup
{
	\let\cleardoublepage\relax
	%%%%%		Nome Indice (NASCOSTO E CREATO A PARTE)
	\renewcommand\contentsname{}
	\begin{tikzpicture}[remember picture, overlay]
		\clip (-80,-95) rectangle (40,10);
		\pgftext[x=.8\textwidth, y=0.2cm]{\HUGE\bfseries 
		Indice}						% Titolo indice
		\end{tikzpicture}
	\vspace{-1cm}
	
	\tableofcontents
	\vspace{.25cm}
}
%%%%%								INTRODUZIONE
		\titleformat{\chapter}
		[hang]
		{\Huge}
		{}
		{0em}
		{}
		[\Large {\begin{tikzpicture} [remember picture, overlay]
		\pgftext[right,x=14.75cm,y=0.2cm]{\HUGE\bfseries 
			前言}
		\end{tikzpicture}}]
%%%%%%%%%%%%%%%%%%%%%%%%%%%%%%%%%%%%%%%%%%%%%%%%%%%%%%%%%%%%%%%%%%%%%%%%%%%%%%%%%
\chapter*{}\normalfont\addcontentsline{toc}{part}{前言}

高效能運算(High-Performance Computing, HPC)的世界最近見證了一個重要里程碑,即在美國橡樹嶺國家實驗室(Oak Ridge National Laboratory)部署的 \textbf{Frontier 超級電腦}首次達成 \textbf{Exascale} 性能。隨著計算性能的此項突破,一類全新的應用場景得以實現,包括:
\begin{itemize}
    \item 天氣與氣候預測,
    \item 生物醫學研究,
    \item 高端設備開發,
    \item 新能源研究與探索,
    \item 動畫設計,
    \item 新材料研究,
    \item 工程設計、模擬與分析,
    \item 遙測數據處理,以及
    \item 金融風險分析。
\end{itemize}

\textbf{AMD} 透過提供一系列高效能的 \textbf{CPU} 和 \textbf{GPU},以及支持 \textbf{HIP} 和 \textbf{ROCm} 執行的開源軟體堆棧,推動了這些進展的實現。這個新興的程式設計生態系統提供了許多創新的功能,包括硬體加速器(如 \textbf{AMD} 和 \textbf{NVIDIA GPU})的互操作性,以及對關鍵高效能編譯器(如 \textbf{LLVM})、叢集部署及核心應用框架(如 \textbf{Raja}、\textbf{Kokkos}、\textbf{TensorFlow} 和 \textbf{PyTorch})的支持,還包含多項高效能函式庫(如 \textbf{rocBLAS}、\textbf{rocSparse}、\textbf{MIOpen}、\textbf{RCCL} 和 \textbf{rocFFT})。

為配合這些進步,高效能運算社群也對此里程碑作出了貢獻,提供了最先進的第三方工具,用於性能監控、除錯器,以及視覺化工具。

由 \textbf{Yifan Sun}、\textbf{Sabila Al Jannat}、\textbf{Trinayan Baruah} 和 \textbf{David Kaeli} 共同撰寫的《\textit{Accelerated Computing with HIP}》第二版,為高效能運算社群提供了一份具參考價值的指南,幫助程式開發人員充分利用 \textbf{Exascale} 計算的優勢。該書內容涵蓋以下主題:


		\titleformat{\chapter}
		[hang]
		{\Huge}
		{}
		{0em}
		{}
		[\Large {\begin{tikzpicture} [remember picture, overlay]
		\pgftext[right,x=14.75cm,y=0.2cm]{\HUGE\bfseries 
			書寫規範}
		\end{tikzpicture}}]

\titlespacing*{\chapter}{0pt}{-30pt}{20pt}

%%%%%%%%%%%%%%%%%%%%%%%%%%%%%%%%%%%%%%%%%%%%%%%%%%%%%%%%%%%%%%%%%%%%%%%%%%%%%%%%%
\chapter*{}\normalfont\addcontentsline{toc}{part}{書寫規範}

% ======================
% 技術書寫規範
% ======================


\section*{文件結構}
\begin{itemize}
    \item 每章使用獨立的 \code{.tex} 文件,主文件透過 \code{\textbackslash input} 引入。
    \item 主文件負責樣式與章節組織,章節文件只包含內容。
\end{itemize}


\section*{字體命令}
\begin{itemize}
    \item \code{\textbackslash code\{example\}}:程式碼 ex: \code{printf}
    \item \code{\textbackslash term\{斜體術語\}}:斜體術語 ex: \term{example}。
    \item \code{\textbackslash bold\{粗體術語\}}:粗體術語 ex: \bold{example}。
\end{itemize}


\section*{引用命令}
\begin{itemize}
    \item \code{\textbackslash figref\{fig:example\}}:引用圖片。
    \item \code{\textbackslash tabref\{tab:example\}}:引用表格。
    \item \code{\textbackslash lstref\{lst:example\}}:引用程式碼塊。
    \item \code{\textbackslash chapref\{chapter\}}:引用 Chapter。
    \item \code{\textbackslash secref\{section\}}:引用 Section。
    
\end{itemize}

\section*{程式碼模板(看左邊 code)}
\begin{lstlisting}[language=Python, caption={範例程式碼:Python 計算平方值}, label={lst:example}]
def square(x):
    return x * x
\end{lstlisting}

\endgroup

%%%%%								ERRATA 
\iffalse
		\titleformat{\chapter}
		[hang]
		{\huge}
		{}
		{0em}
		{}
		[\large {\begin{tikzpicture} [remember picture, overlay]
		\pgftext[right,x=14.75cm,y=0.2cm]{\color{black}\Huge\bfseries 
			Errata corrige \& Aggiunte};
		\end{tikzpicture}}]
\chapter*{}\normalfont		\addcontentsline{toc}{part}{Errata corrige \& Aggiunte}
\begin{longtable}{p{2.55cm}p{1.45cm}p{9cm}}
	Data di \newline correzione & Pagina&\\\hline
	24/10/2023	& ?	& ?
\end{longtable}
\fi

%%%%%%%%%%%%%%%%%%%%%%%%%%%%									MAINMATTER
%%%%%%%%%%%%%%%%%%%%%%%%%%%%
\mainmatter

\pagestyle{fancymain}
\titleformat{\chapter}[display]{\bfseries\Large}	{\filleft\MakeUppercase{\chaptertitlename} \HUGE\thechapter}{.5ex}{\titlerule\vspace{1ex}\filleft}[\vspace{3.5ex}]
\titlespacing*{\chapter}{0pt}{0.1\baselineskip}{0.5\baselineskip}

\fancyheadoffset[L]{\dimexpr\oddsidemargin-0in\relax}
\fancyheadoffset[R]{\dimexpr\oddsidemargin-0in\relax}

\normalfont
\normalsize


% \newgeometry{top=35mm, bottom=35mm, left=15mm, right=15mm, headheight=0pt, headsep=0pt, marginparsep=0pt, marginparwidth=0pt, footskip=0pt, footnotesep=0pt}
% % \part*{\HUGE Parte 1}\label{Parte1}
% \restoregeometry

%%%%%								CAPITOLI
\input{Chapters/Chap1}
\chapter{HIP 程式設計入門}
\section{fill your section title here}


%%%%%								APPENDICI
\newgeometry{top=35mm, bottom=35mm, left=15mm, right=15mm, headheight=0pt, headsep=0pt, marginparsep=0pt, marginparwidth=0pt, footskip=0pt, footnotesep=0pt}
\part*{\HUGE Appendici}
\titleformat{\chapter}[display]    	{\bfseries\large\raggedright}    	{\vspace{-2.35cm} \MakeUppercase{\chaptertitlename}\ \Huge \thechapter}    	{.125ex}    	{\raggedleft\vspace{-1cm}\Huge\makebox[.5\textwidth]{}}
\titlespacing*{\chapter}{0pt}{6\baselineskip}{2.5\baselineskip}
\restoregeometry

	% Capitoli
\titlecontents{chapter}[2.5pc]
{\addvspace{15pt}}
{\begin{tikzpicture}
		\pgftext{\LArge\bfseries\bfseries\color{black}\hspace{-1cm} Appendice\ \thecontentslabel{\color{white}.}\hspace{.5cm} }
	\end{tikzpicture}\Large }
{}
{\color{black}\titlerule\; \;\Large\bfseries Pagina \thecontentspage}

\pagestyle{fancyapp}

\begin{appendices}
	\input{Appendici/AppendiceA}

%%%%%%%%%%%%%%%%%%%%%%%%%%%%									BACKMATTER
%%%%%%%%%%%%%%%%%%%%%%%%%%%%
\backmatter

%%%%% 							BIBLIOGRAFIA
\pagestyle{fancyBibliografia}
\titleformat{\chapter}
	[hang]
	{\vspace{-2cm}\Huge}
	{}
	{0em}
	{}
	[\Large {\begin{tikzpicture} [remember picture, overlay]
	\pgftext[right,x=14.75cm,y=0.2cm]{\HUGE\bfseries 
	Bibliografia}
	\end{tikzpicture}}]
	
	\nocite{*}
	\bibliographystyle{amsalpha}
	\bibliography{FileAusiliari/Bibliografia}
	\addcontentsline{toc}{part}{Bibliografia}
\cleardoublepage
%INDICE ANALITICO
 \pagestyle{fancyIndiceAnalitico}
 	\renewcommand{\indexname}{}
	% SISTEMA IL PROBLEMA DEL LINK ALL'INDICE ANALITICO
	\let\cleardoublepage\relax
	\titleformat{\chapter}[hang]{}{}{0em}{}[]
 	\chapter*{}
	\titleformat{\chapter}
		[hang]
		{\Huge}
		{}
		{0em}
		{}
		[\Large {\begin{tikzpicture} [remember picture, overlay]
		\pgftext[right,x=14.75cm,y=0.2cm]{\HUGE\bfseries 
			Indice analitico}
		\end{tikzpicture}}]
	\titlespacing*{\chapter}{0pt}{0\baselineskip}{5\baselineskip}
	\addcontentsline{toc}{part}{Indice analitico}	
	\vspace{-2cm}
	\printindex
%%%%%%%%%%%%%%%%%%%%%%%%%%%%%%%%%%%%%%%%%%%%%%%%%%%%%%%%%%%%%%%%%%%%%%%%%%%%%%%%%%%%%%%%%%%%%%%%%%%%%%%%%%%%%%%%%
\end{appendices}
\end{document}
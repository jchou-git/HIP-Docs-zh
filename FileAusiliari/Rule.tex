		\titleformat{\chapter}
		[hang]
		{\Huge}
		{}
		{0em}
		{}
		[\Large {\begin{tikzpicture} [remember picture, overlay]
		\pgftext[right,x=14.75cm,y=0.2cm]{\HUGE\bfseries 
			書寫規範}
		\end{tikzpicture}}]

\titlespacing*{\chapter}{0pt}{-30pt}{20pt}

%%%%%%%%%%%%%%%%%%%%%%%%%%%%%%%%%%%%%%%%%%%%%%%%%%%%%%%%%%%%%%%%%%%%%%%%%%%%%%%%%
\chapter*{}\normalfont\addcontentsline{toc}{part}{書寫規範}

% ======================
% 技術書寫規範
% ======================


\section*{文件結構}
\begin{itemize}
    \item 每章使用獨立的 \code{.tex} 文件,主文件透過 \code{\textbackslash input} 引入。
    \item 主文件負責樣式與章節組織,章節文件只包含內容。
\end{itemize}


\section*{字體命令}
\begin{itemize}
    \item \code{\textbackslash code\{example\}}:程式碼 ex: \code{printf}
    \item \code{\textbackslash term\{斜體術語\}}:斜體術語 ex: \term{example}。
    \item \code{\textbackslash bold\{粗體術語\}}:粗體術語 ex: \bold{example}。
\end{itemize}


\section*{引用命令}
\begin{itemize}
    \item \code{\textbackslash figref\{fig:example\}}:引用圖片。
    \item \code{\textbackslash tabref\{tab:example\}}:引用表格。
    \item \code{\textbackslash lstref\{lst:example\}}:引用程式碼塊。
    \item \code{\textbackslash chapref\{chapter\}}:引用 Chapter。
    \item \code{\textbackslash secref\{section\}}:引用 Section。
    
\end{itemize}

\section*{程式碼模板(看左邊 code)}
\begin{lstlisting}[language=Python, caption={範例程式碼:Python 計算平方值}, label={lst:example}]
def square(x):
    return x * x
\end{lstlisting}

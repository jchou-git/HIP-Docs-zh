		\titleformat{\chapter}
		[hang]
		{\Huge}
		{}
		{0em}
		{}
		[\Large {\begin{tikzpicture} [remember picture, overlay]
		\pgftext[right,x=14.75cm,y=0.2cm]{\HUGE\bfseries 
			前言}
		\end{tikzpicture}}]
%%%%%%%%%%%%%%%%%%%%%%%%%%%%%%%%%%%%%%%%%%%%%%%%%%%%%%%%%%%%%%%%%%%%%%%%%%%%%%%%%
\chapter*{}\normalfont\addcontentsline{toc}{part}{前言}

高效能運算(High-Performance Computing, HPC)的世界最近見證了一個重要里程碑,即在美國橡樹嶺國家實驗室(Oak Ridge National Laboratory)部署的 \textbf{Frontier 超級電腦}首次達成 \textbf{Exascale} 性能。隨著計算性能的此項突破,一類全新的應用場景得以實現,包括:
\begin{itemize}
    \item 天氣與氣候預測,
    \item 生物醫學研究,
    \item 高端設備開發,
    \item 新能源研究與探索,
    \item 動畫設計,
    \item 新材料研究,
    \item 工程設計、模擬與分析,
    \item 遙測數據處理,以及
    \item 金融風險分析。
\end{itemize}

\textbf{AMD} 透過提供一系列高效能的 \textbf{CPU} 和 \textbf{GPU},以及支持 \textbf{HIP} 和 \textbf{ROCm} 執行的開源軟體堆棧,推動了這些進展的實現。這個新興的程式設計生態系統提供了許多創新的功能,包括硬體加速器(如 \textbf{AMD} 和 \textbf{NVIDIA GPU})的互操作性,以及對關鍵高效能編譯器(如 \textbf{LLVM})、叢集部署及核心應用框架(如 \textbf{Raja}、\textbf{Kokkos}、\textbf{TensorFlow} 和 \textbf{PyTorch})的支持,還包含多項高效能函式庫(如 \textbf{rocBLAS}、\textbf{rocSparse}、\textbf{MIOpen}、\textbf{RCCL} 和 \textbf{rocFFT})。

為配合這些進步,高效能運算社群也對此里程碑作出了貢獻,提供了最先進的第三方工具,用於性能監控、除錯器,以及視覺化工具。

由 \textbf{Yifan Sun}、\textbf{Sabila Al Jannat}、\textbf{Trinayan Baruah} 和 \textbf{David Kaeli} 共同撰寫的《\textit{Accelerated Computing with HIP}》第二版,為高效能運算社群提供了一份具參考價值的指南,幫助程式開發人員充分利用 \textbf{Exascale} 計算的優勢。該書內容涵蓋以下主題:

